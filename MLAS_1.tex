\documentclass[twocolumn]{article}
\usepackage{graphicx} % Required for inserting images

\title{ECG Heartbeat Categorization Dataset}
\author{Nguyen Hoang Duong}
\date{February 2024}

\begin{document}

\maketitle

\section{Introduction}
Heart is the most important viscera in a person's body, is a hollow muscular organ that pumps the blood through the circulatory system by rhythmic contraction and dilation. By initiating the rhythmic pulsation known as the heartbeat, which is the regular movement of the heart as it pumps blood around human body for facilitating the circulation of oxygen-rich blood throughout the body. Heartbeat can be measurable so it's present the pumping iteration, normally from 60 to 100 beats per minutes for a normal human. The lower, the better heart function and better cardiovascular fitness. From that, it's lead to heartbeat prediction job, known as heartbeat in clinical diagnosis. Heartbeat in clinical diagnosis refers to the assessment of the rhythmic contractions and relaxations of the heart muscle, providing essential information about cardiac function and overall cardiovascular health. 
This evaluation is vital for diagnosing various heart conditions such as arrhythmias, heart murmurs, and heart failure, allowing healthcare professionals to determine the heart's rate, rhythm, and strength. Monitoring the heartbeat assists in early detection, accurate diagnosis, and effective management of cardiac disorders, ultimately improving patient outcomes and quality of life.
\section{Background}
ECG Heartbeat Categorization Dataset
This dataset is composed of two collections of heartbeat signals derived from two famous datasets in heartbeat classification, the MIT - BIH Arrhythmia Dataset and The PTB Diagnostic ECG Database. The number of samples in both collections is large enough for training a deep neural network

This dataset has been used in exploring heartbeat classification using deep neural network architectures, and observing some of the capabilities of transfer learning on it. The signals correspond to electrocardiogram (ECG) shapes of heartbeats for the normal case and the cases affected by different arrhythmias and myocardial infarction. These signals are preprocessed and segmented, with each segment corresponding to a heartbeat

The MIT - BIH dataset consists of ECG recording from 47 different subjects recorded at the sampling rate of 360 Hz. Each beat is annotated by at least cardiologists. We use annotations in this dataset to create five different beat categories in accordance with Association for the Advancement of Medical Instrumentation EC57 standard.

The PTB Diagnostics dataset consists of ECG records from 290 subjects: 148 diagnosed as MI , 52 healthy control, and the rest are diagnosed with 7 different disease. Each record contains ECG signals from 12 leads sampled at the frequency of 1000Hz

A Convolutional Neural Network (CNN) is a type of Deep Learning neural network architecture commonly used in Computer Vision. Computer vision is a field of Artificial Intelligence that enables a computer to understand and interpret the image or visual data. 
When it comes to Machine Learning, Artificial Neural Networks perform really well. Neural Networks are used in various datasets like images, audio, and text. Different types of Neural Networks are used for different purposes, for example for predicting the sequence of data in this case is heartbeat sequence.
\section{Method}
In this paper we suggest training a convolutional neural network for classification of ECG beat types on the MIT-BIH dataset. The trained network not only can be used for the purpose of beat classification, but also in the next section we show that it can be used as an informative representation of heartbeats.

illustrates the network architecture proposed for the beat classification task. Here, all convolution layers are applying 1-D convolution through time and each have 32 kernels of size 5. We also use max pooling of size 5 and stride 2 in all pooling layers. The predictor network consists of five residual blocks followed by two fully-connected layers with 32 neurons each and a softmax layer to predict output class

Confusion matrix for heartbeat classification on the test set. Total number of samples in each class is indicated inside the parenthesis. Number inside blocks are number of samples classified in each category normalized by total number of samples and round to two digits probabilities. Each residual block contains two convolutional layers, two ReLu nonlinearities, a residual skip connection, and a pooling layer. In total, the resulting network is a deep network consisting of 13 weight layers

After training the network, we use the output activations of the very last convolution as a representation of input beats. Here, we use this representation as input to a two layer fully-connected network with 32 neurons at each layer to predict MI. It is noteworthy to mention that during the training for the MI prediction task, we freeze the weights for all other layers aside from the last two

In all experiments, Pytorch computational library is used for model training and evaluation. Cross entropy loss on the softmax outputs is used as the loss function. For training the networks, we used Stochastic gradient descent optimization method with the learning rate equal to 0.01 for 10 iterations

\section{Evaluation}
After training the model for 10 iterations, we got the train loss stay remaining at iteration 2 to the end, accuracy of training at 78.77. When performing on the test set, the test loss stay remaining at iteration 1 to the end with the accuracy of testing at 75.12

We evaluated the arrhythmia classifer on 4079 heartbeats (about 819 from each class) that are not used in the network training phase. Presents the confusion matrix of applying the classifier on the test set. As it can be seen from this figure, the model is able to make accurate predictions and distinguish different classes

Based on the training and testing loss, it's seem to be that the model didn't learn anything maybe cause by the dataset which contain many unnecessary value
\section{Conclusion}
To cover all the report, we built a classification model for the heatbeat signal with ECG Heartbeat Category Dataset. According to the result, we should do more preprocessing step first to obtain a better data value before the model handling it and visualize the signal type for understanding.
\end{document}
